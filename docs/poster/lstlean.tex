% Listing style definition for the Lean Theorem Prover.
% Defined by Jeremy Avigad, 2015, by modifying Assia Mahboubi's SSR style.
% Unicode replacements taken from Olivier Verdier's unixode.sty

\lstdefinelanguage{lean} {

% Anything betweeen $ becomes LaTeX math mode
mathescape=false,
% Comments may or not include Latex commands
texcl=false,

% keywords, list taken from lean-syntax.el
morekeywords=[1]{
import, prelude,
open, as, renaming, replacing, hiding, exposing, export,
namespace, section,
parameter, parameters, variable, variables, universe, universes, include, omit,
protected, private, noncomputable, meta, mutual, theory,
definition, def, constant, constants, lemma, theorem, example, axiom, axioms,
inductive, structure, class, extends,
begin, end, match, calc, this, with, have, show, suffices, by, in, at, let, forall, Pi, fun,
exists, if, dif, then, else, assume, from, to, do,
using, using_well_founded,
instance, attribute,
precedence, infix, infixl, infixr, notation, postfix, prefix,
reserve, local,
set_option, run_command,
alias, declare_trace, add_key_equivalence, aliases, register_simp_ext,
help, print, eval, check},

% Sorts
morekeywords=[2]{Sort, Type, Prop, Type*, Type₀, Type₁, Type₂, Type₃},

% Errors
morekeywords=[3]{sorry, admit},

% tactics, list taken from lean-syntax.el
% morekeywords=[4]{
% Cond, or_else, then, try, when, assumption, eassumption, rapply,
% apply, fapply, eapply, rename, intro, intros, rintro, rintros, all_goals, fold, focus, focus_at,
% generalize, generalizes, clear, clears, revert, reverts, back, beta, done, exact, rexact,
% refine, repeat, whnf, rotate, rotate_left, rotate_right, inversion, cases, rcases, rewrite,
% xrewrite, krewrite, blast, simp, esimp, simpa, unfold, change, check_expr, contradiction,
% exfalso, split, existsi, constructor, fconstructor, left, right, injection, congruence, reflexivity,
% symmetry, transitivity, state, induction, induction_using, fail, append,
% substvars, now, with_options, with_attributes, with_attrs, note
% norm_num, norm_cast, lift, library_search, suggest, abel, ring, linarith, omega, tidy, finish,
% solve_by_elim, clarify, safe,
% failed, infer_type, unify, return, local_context, target
% },

% modifiers, taken from lean-syntax.el
% note: 'otherkeywords' is needed because these use a different symbol.
% this command doesn't allow us to specify a number -- they are put with [1]
% otherkeywords={
% [persistent], [notation], [visible], [instance], [trans_instance],
% [class], [parsing-only], [coercion], [unfold_full], [constructor],
% [reducible], [irreducible], [semireducible], [quasireducible], [wf],
% [whnf], [multiple_instances], [none], [decl], [declaration],
% [relation], [symm], [subst], [refl], [trans], [simp], [simps], [congr], [unify],
% [backward], [forward], [no_pattern], [begin_end], [tactic], [abbreviation],
% [reducible], [unfold], [alias], [eqv], [intro], [intro!], [elim], [grinder],
% [localrefinfo], [recursor], @
% },

% Various symbols
literate=
{α}{{\ensuremath{\mathrm{\upalpha}}}}1
{β}{{\ensuremath{\mathrm{\upbeta}}}}1
{γ}{{\ensuremath{\mathrm{\upgamma}}}}1
{δ}{{\ensuremath{\mathrm{\updelta}}}}1
{ε}{{\ensuremath{\mathrm{\varepsilon}}}}1
{ζ}{{\ensuremath{\mathrm{\zeta}}}}1
{η}{{\ensuremath{\mathrm{\eta}}}}1
{θ}{{\ensuremath{\mathrm{\theta}}}}1
{ι}{{\ensuremath{\mathrm{\iota}}}}1
{κ}{{\ensuremath{\mathrm{\kappa}}}}1
{μ}{{\ensuremath{\mathrm{\mu}}}}1
{ν}{{\ensuremath{\mathrm{\nu}}}}1
{ξ}{{\ensuremath{\mathrm{\xi}}}}1
{π}{{\ensuremath{\mathrm{\mathnormal{\pi}}}}}1
{ρ}{{\ensuremath{\mathrm{\rho}}}}1
{σ}{{\ensuremath{\mathrm{\sigma}}}}1
{τ}{{\ensuremath{\mathrm{\tau}}}}1
{φ}{{\ensuremath{\mathrm{\varphi}}}}1
{χ}{{\ensuremath{\mathrm{\chi}}}}1
{ψ}{{\ensuremath{\mathrm{\psi}}}}1
{ω}{{\ensuremath{\mathrm{\omega}}}}1

{Γ}{{\ensuremath{\mathrm{\Gamma}}}}1
{Δ}{{\ensuremath{\mathrm{\Delta}}}}1
{Θ}{{\ensuremath{\mathrm{\Theta}}}}1
{Λ}{{\ensuremath{\mathrm{\Lambda}}}}1
{Σ}{{\ensuremath{\mathrm{\Sigma}}}}1
{Φ}{{\ensuremath{\mathrm{\Phi}}}}1
{Ξ}{{\ensuremath{\mathrm{\Xi}}}}1
{Ψ}{{\ensuremath{\mathrm{\Psi}}}}1
{Ω}{{\ensuremath{\mathrm{\Omega}}}}1

{ℵ}{{\ensuremath{\aleph}}}1

{≤}{{\color{symbolcolor}\ensuremath{\leq}}}1
{≥}{{\color{symbolcolor}\ensuremath{\geq}}}1
{≠}{{\color{symbolcolor}\ensuremath{\neq}}}1
{≈}{{\color{symbolcolor}\ensuremath{\approx}}}1
{≡}{{\color{symbolcolor}\ensuremath{\equiv}}}1
{≃}{{\color{symbolcolor}\ensuremath{\simeq}}}1

{∂}{{\color{symbolcolor}\ensuremath{\partial}}}1
{∆}{{\color{symbolcolor}\ensuremath{\triangle}}}1 % or \laplace?

{∫}{{\color{symbolcolor}\ensuremath{\int}}}1
{∑}{{\color{symbolcolor}\ensuremath{\mathrm{\Sigma}}}}1
%{Π}{{\ensuremath{\mathrm{\Pi}}}}1

{⊥}{{\color{symbolcolor}\ensuremath{\perp}}}1
{∞}{{\color{symbolcolor}\ensuremath{\infty}}}1
{∂}{{\color{symbolcolor}\ensuremath{\partial}}}1

{∓}{{\color{symbolcolor}\ensuremath{\mp}}}1
{±}{{\color{symbolcolor}\ensuremath{\pm}}}1
{×}{{\color{symbolcolor}\ensuremath{\times}}}1

{⊕}{{\color{symbolcolor}\ensuremath{\oplus}}}1
{⊗}{{\color{symbolcolor}\ensuremath{\otimes}}}1
{⊞}{{\color{symbolcolor}\ensuremath{\boxplus}}}1

{∇}{{\color{symbolcolor}\ensuremath{\nabla}}}1
{√}{{\color{symbolcolor}\ensuremath{\sqrt}}}1

{⬝}{{\color{symbolcolor}\ensuremath{\cdot}}}1
{•}{{\color{symbolcolor}\ensuremath{\cdot}}}1
{∘}{{\color{symbolcolor}\ensuremath{\circ}}}1
{`}{{\ensuremath{{}^\backprime}}}1
{'}{{\ensuremath{{}^\prime}}}1

%{⁻}{{\ensuremath{^{\textup{\kern1pt\rule{2pt}{0.3pt}\kern-1pt}}}}}1
{⁻}{{\ensuremath{^{-}}}}1
{▸}{{\ensuremath{\blacktriangleright}}}1

{∧}{{\color{symbolcolor}\ensuremath{\wedge}}}1
{∨}{{\color{symbolcolor}\ensuremath{\vee}}}1
{¬}{{\color{symbolcolor}\ensuremath{\neg}}}1
{⊢}{{\color{symbolcolor}\ensuremath{\vdash}}}1

%{⟨}{{\ensuremath{\left\langle}}}1
%{⟩}{{\ensuremath{\right\rangle}}}1
{⟨}{{\ensuremath{\langle}}}1
{⟩}{{\ensuremath{\rangle}}}1

{↦}{{\color{symbolcolor}\ensuremath{\mapsto}}}1
{→}{{\color{symbolcolor}\ensuremath{\rightarrow}}}1
{←}{{\color{symbolcolor}\ensuremath{\leftarrow}}}1
{↔}{{\color{symbolcolor}\ensuremath{\leftrightarrow}}}1
{⇒}{{\color{symbolcolor}\ensuremath{\Rightarrow}}}1
{⟹}{{\color{symbolcolor}\ensuremath{\Longrightarrow}}}1
{⇐}{{\color{symbolcolor}\ensuremath{\Leftarrow}}}1
{⟸}{{\color{symbolcolor}\ensuremath{\Longleftarrow}}}1

{∩}{{\color{symbolcolor}\ensuremath{\cap}}}1
{∪}{{\color{symbolcolor}\ensuremath{\cup}}}1
{⊂}{{\color{symbolcolor}\ensuremath{\subseteq}}}1
{⊆}{{\color{symbolcolor}\ensuremath{\subseteq}}}1
{⊄}{{\color{symbolcolor}\ensuremath{\nsubseteq}}}1
{⊈}{{\color{symbolcolor}\ensuremath{\nsubseteq}}}1
{⊃}{{\color{symbolcolor}\ensuremath{\supseteq}}}1
{⊇}{{\color{symbolcolor}\ensuremath{\supseteq}}}1
{⊅}{{\color{symbolcolor}\ensuremath{\nsupseteq}}}1
{⊉}{{\color{symbolcolor}\ensuremath{\nsupseteq}}}1
{∈}{{\color{symbolcolor}\ensuremath{\in}}}1
{∉}{{\color{symbolcolor}\ensuremath{\notin}}}1
{∋}{{\color{symbolcolor}\ensuremath{\ni}}}1
{∌}{{\color{symbolcolor}\ensuremath{\notni}}}1
{∅}{{\color{symbolcolor}\ensuremath{\emptyset}}}1

{∖}{{\color{symbolcolor}\ensuremath{\setminus}}}1
{†}{{\color{symbolcolor}\ensuremath{\dag}}}1

{ℕ}{{\ensuremath{\mathbb{N}}}}1
{ℤ}{{\ensuremath{\mathbb{Z}}}}1
{ℝ}{{\ensuremath{\mathbb{R}}}}1
{ℚ}{{\ensuremath{\mathbb{Q}}}}1
{ℂ}{{\ensuremath{\mathbb{C}}}}1
{⌞}{{\ensuremath{\llcorner}}}1
{⌟}{{\ensuremath{\lrcorner}}}1
{⦃}{{\ensuremath{\{\!|}}}1
{⦄}{{\ensuremath{|\!\}}}}1
{∥}{{\ensuremath{\|}}}1

{₁}{{\ensuremath{_1}}}1
{₂}{{\ensuremath{_2}}}1
{₃}{{\ensuremath{_3}}}1
{₄}{{\ensuremath{_4}}}1
{₅}{{\ensuremath{_5}}}1
{₆}{{\ensuremath{_6}}}1
{₇}{{\ensuremath{_7}}}1
{₈}{{\ensuremath{_8}}}1
{₉}{{\ensuremath{_9}}}1
{₀}{{\ensuremath{_0}}}1
{ₐ}{{\ensuremath{_a}}}1
{ᵢ}{{\ensuremath{_i}}}1
{ⱼ}{{\ensuremath{_j}}}1
{ₙ}{{\ensuremath{_n}}}1
{ₘ}{{\ensuremath{_m}}}1

{¹}{{\ensuremath{^1}}}1
{ᵒᵖ}{{\color{symbolcolor}\textsuperscript{op}}}1

{↑}{{\color{symbolcolor}\ensuremath{\uparrow}}}1
{↓}{{\color{symbolcolor}\ensuremath{\downarrow}}}1
{⟶}{{\color{symbolcolor}\ensuremath{\longrightarrow}}}1
{⥤}{{\color{symbolcolor}\ensuremath{\Rightarrow}}}1

{...}{{\ensuremath{\ldots}}}1

{▸}{{\ensuremath{\triangleright}}}1

{𝒳}{{\ensuremath{\mathcal{X}}}}1

{⌊}{{\ensuremath{\lfloor}}}1
{⌋}{{\ensuremath{\rfloor}}}1
{⌈}{{\ensuremath{\lceil}}}1
{⌉}{{\ensuremath{\rceil}}}1

{Σ}{{\color{symbolcolor}\ensuremath{\Sigma}}}1
{Π}{{\color{symbolcolor}\ensuremath{\Uppi}}}1 %\mathrm{\Uppi}
{∀}{{\color{symbolcolor}\ensuremath{\forall}}}1
{∃}{{\color{symbolcolor}\ensuremath{\exists}}}1
{λ}{{\color{symbolcolor}\ensuremath{\uplambda}}}1
{\$}{{\color{symbolcolor}\$}}1

{:}{{\color{symbolcolor}:}}1
{|}{{\color{symbolcolor}|}}1
{=}{{\color{symbolcolor}=}}1
{<}{{\color{symbolcolor}<}}1
{>}{{\color{symbolcolor}>}}1
{<|>}{{\color{symbolcolor}<|>}}1
{+}{{\color{symbolcolor}+}}1
{*}{{\color{symbolcolor}\ensuremath{{}^{*}}}}1
{\#}{{\color{keywordcolor}\#}}1, % is usually part of a keyword, like #check

% Comments
%comment=[s][\itshape \color{commentcolor}]{/-}{-/},
morecomment=[s]{/-}{-/},
morecomment=[l]{--},

% Spaces are not displayed as a special character
showstringspaces=false,

% keep spaces
keepspaces=true,

% String delimiters
morestring=[b]{"},

% Size of tabulations
tabsize=3,

% Enables ASCII chars 128 to 255
extendedchars=false,

% Case sensitivity
sensitive=true,

% Automatic breaking of long lines
breaklines=true,
breakatwhitespace=true,

% Default style fors listingsred
lineskip={-1.5pt},
basicstyle={\ttfamily},
% Position of captions is bottom
captionpos=b,

% Full flexible columns
columns=[l]fullflexible,


% Style for (listings') identifiers
identifierstyle={\ttfamily\color{black}},
% Note : highlighting of Coq identifiers is done through a new
% delimiter definition through an lstset at the begining of the
% document. Don't know how to do better.

% Style for declaration keywords
keywordstyle=[1]{\ttfamily\color{keywordcolor}},

% Style for sorts
keywordstyle=[2]{\ttfamily\color{sortcolor}},

% Style for errors
keywordstyle=[3]{\ttfamily\color{errorcolor}},

% Style for tactics keywords
% keywordstyle=[4]{\ttfamily\color{tacticcolor}},

% Style for attributes
% keywordstyle=[5]{\ttfamily\color{attributecolor}},

% Style for strings
stringstyle={\ttfamily\color{stringcolor}},

% Style for comments
commentstyle={\ttfamily\itshape\color{commentcolor}},

}